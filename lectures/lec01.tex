\chapter{Gruppen}\lecture

\section{Gruppen und Gruppenhomomorphismen}

\textbf{Motivation:} aus dem ersten Jahr kennen wir viele Gruppen, z.B. $(\R,+), (\Z,+), \Z/m\Z$ für $m \in \N$, $\R^n, S_n = $ Permutationen auf $n$ Elemente, Funktionen $f: \R \to \C$ mit punktweiser Addition\\
\textbf{erstes Ziel:}
    \begin{itemize}
        \item Wiederholung Grundbegriffe von Gruppen
        \item erste Resultate zur Theorie endlicher Gruppen
    \end{itemize}

\begin{defn*}[Monoid]\index{Monoid}
    Ein Monoid ist eine Menge $M$ zusammen mit einer Verknüpfung $\circ: M \times M \to M$, die folgende Eigenschaften erfüllt:
    \begin{enumerate}[label={\roman*})]
        \item $\foralll a,b,c \in M$ gilt $(a \circ b) \circ c = a \circ (b \circ c)$
        \item es gibt ein Einselement $e \in M$ mit $e \circ a = a = a \circ e \ \foralll a \in M$
    \end{enumerate}
\end{defn*}

\begin{rem*}
	$ (\N_{\geq 0},+),(\Q_{\geq 0},+) $ sind Monoide aber keine Gruppe.
\end{rem*}

\begin{defn*}[Inverselemente]\index{Inverselemente}
	Sei $ (M,\circ) $ ein Monoid und $a \in M$. Wir nennen $b$ invers zu $a$/\emph{inverses Element} zu $a$, falls $b \circ a = a \circ b = e$.
\end{defn*}

\begin{rem*}
	Sind $b, b' \in M$ invers zu $a$, dann ist $b = b'$, denn $b = b \circ e = b \circ (a \circ b') = (b \circ a) \circ b' = e \circ b' = b'$
\end{rem*}

\begin{exmp*}
	Im $ (\N_{\geq 0},+) $ ist 0 das einzige Element, das ein inverses Element hat.
\end{exmp*}

\underline{Notation:} Ist $ a \in M $ und $ b \in M $ invers zu $a$, so schreiben wir $ b = a^{-1} $.

\begin{defn*}[Gruppe]\index{Gruppe}
	Wir nennen ein Monoid $ (G,\circ) $ eine \emph{Gruppe}, falls jedes $ a \in G $ ein inverses Element $ a^{-1} \in G $ besitzt.
\end{defn*}

\begin{exmp*}
	$ GL_n(\R) = \{A \in M_{n \times n} (\R) \mid \det(A) \neq 0, n \geq 0\} $ ist eine Gruppe unter Matrixmultiplikation. Für $ n \geq 2 $ gibt es Matrizen $ A,B \in GL_n(\R) $ mit $ AB \neq BA $.
\end{exmp*}

\begin{defn*}[abelsche Gruppe]\index{Gruppe!abelsche}
	Sei $ (G,\circ) $ eine Gruppe. $G$ heißt \emph{kommutativ} oder \emph{abelsch}, falls gilt $ a \circ b = b \circ a \ \foralll a,b \in G $.
\end{defn*}

\begin{exmp*}
	\begin{enumerate}[label=\textcircled{\alph*}]
		\item Die Gruppe aller Diagonalmatrizen in $ GL_n(\R) $ 
			$$ \left\{A \in GL_n(\R) \mid A = \begin{pmatrix}
				\alpha_1 & & 0 \\ & \ddots & \\ 0 & & \alpha_n
			\end{pmatrix}, \alpha_1, \dotsc, \alpha_n \in \R \setminus\{0\} \right\} $$
			ist eine kommutative Gruppe unter Matrixmultiplikation.
		\item $ M_{n \times n}(\R) $ ist eine abelsche Gruppe unter Addition von Matrizen mit neutralem Element $ \begin{pmatrix}
			0 & \dots & 0 \\ \vdots & \ddots & \vdots \\ 0 & \dots & 0
		\end{pmatrix} $
	\end{enumerate}
\end{exmp*}

\begin{rem*}
	Sei $ I $ eine Indexmenge und $ G_i, i \in I, $ Gruppen. Dann ist $ \prod_{i \in I} G_i $ wieder eine Gruppe unter der Verknüpfung $ \big((g_i)_{i \in I}, (h_i)_{i \in I}\big) \mapsto (\underbrace{g_i h_i}_{\in G_i})_{i \in I} $ für $ g_i,h_i \in G_i, i \in I $.
\end{rem*}

\begin{exmp*}
	Für $ m \in N $ ist $ \Z / m\Z $ eine Gruppe unter Addition. Wir können nach dieser Bemerkung daraus (endliche abelsche) Gruppen
	\[ \Z/m_1\Z \times \Z/m_2\Z \times \dots \times \Z/m_n\Z \]
	für $ m_1, \dotsc, m_n \in N $ konstruieren.
\end{exmp*}

\begin{defn*}[Untermonoid, Untergruppe] \index{Untermonoid} \index{Untergruppe}
	Sei $M$ ein Monoid und $ H \subseteq M $. Wir nennen $H$ ein \emph{Untermonoid}, falls $ e \in H $ und gilt $ a,b \in H \implies a \circ b \in H $. Sei $G$ eine Gruppe. Eine Teilmenge $ \emptyset \neq H \subseteq G $ heißt \emph{Untergruppe} von $G$, falls gilt $ \foralll a,b \in H: a \circ b^{-1} \in H $.\\
	\underline{Notation:} Ist $H$ eine Untergruppe von $G$, so schreibe auch $H \leq G$ und $H < G$ falls $ H \neq G $.
\end{defn*}

\begin{exmp*}
	\begin{enumerate}[label=\textcircled{\alph*}]
		\item Für eine beliebige Gruppe $G$ sind $\{e\}$ und $G$ stets Untergruppen von $G$.
		\item Sei $ (G,\circ) = (\Z,+) $ und $m \in \N$. Dann ist $ m \cdot \Z \subseteq \Z $ eine Untergruppe von $\Z$ und $ m \cdot \Z_{geq 0} $ ein Untermonoid von $\Z$.
		\item $ (\Z,+) $ ist eine Untergruppe $ (\C,+) $, $(\Z_{geq 0},+)$ ist Untermonoid von $(\C,+)$.
		\item Sei $\K$ ein Körper und $n \in \N$. Dann ist $ SL_n(\K) = \{A \in Mat_{n \times n}(\K) \mid \det(A) = 1\} $ eine Untergruppe von $ GL_n(\K) = \{A \in M_{n \times n}(\K) \mid \det(A) \neq 0\} $.
	\end{enumerate}
\end{exmp*}

\begin{rem*}
	Sei $G$ eine Gruppe und $ H_i, i \in I $ Untergruppen von $G$. Dann ist auch $ \bigcap_{i \in I} H_i $ eine Untergruppe von $G$.
\end{rem*}