\chapter{Gruppen}\lecture

\section{Gruppen und Gruppenhomomorphismen}

\textbf{Motivation:} aus dem ersten Jahr kennen wir viele Gruppen, z.B. $(\R,+), (\Z,+), \Z/m\Z$ für $m \in \N$, $\R^n, S_n = $ Permutationen auf $n$ Elemente, Funktionen $f: \R \to \C$ mit punktweiser Addition\\
\textbf{erstes Ziel:}
    \begin{itemize}
        \item Wiederholung Grundbegriffe von Gruppen
        \item erste Resultate zur Theorie endlicher Gruppen
    \end{itemize}

\begin{defn*}[Monoid]\index{Monoid}
    Ein Monoid ist eine Menge $M$ zusammen mit einer Verknüpfung $\circ: M \times M \to M$, die folgende Eigenschaften erfüllt:
    \begin{enumerate}[label={\roman*})]
        \item $\foralll a,b,c \in M$ gilt $(a \circ b) \circ c = a \circ (b \circ c)$
        \item es gibt ein Einselement $e \in M$ mit $e \circ a = a = a \circ e \ \foralll a \in M$
    \end{enumerate}
\end{defn*}

\begin{rem*}
	$ (\N_{\geq 0},+),(\Q_{\geq 0},+) $ sind Monoide aber keine Gruppe.
\end{rem*}

\begin{defn*}[Inverselemente]\index{Inverselemente}
	Sei $ (M,\circ) $ ein Monoid und $a \in M$. Wir nennen $b$ invers zu $a$/\emph{inverses Element} zu $a$, falls $b \circ a = a \circ b = e$.
\end{defn*}

\begin{rem*}
	Sind $b, b' \in M$ invers zu $a$, dann ist $b = b'$, denn $b = b \circ e = b \circ (a \circ b') = (b \circ a) \circ b' = e \circ b' = b'$
\end{rem*}

\begin{exmp*}
	Im $ (\N_{\geq 0},+) $ ist 0 das einzige Element, das ein inverses Element hat.
\end{exmp*}

\underline{Notation:} Ist $ a \in M $ und $ b \in M $ invers zu $a$, so schreiben wir $ b = a^{-1} $.

\begin{defn*}[Gruppe]\index{Gruppe}
	Wir nennen ein Monoid $ (G,\circ) $ eine \emph{Gruppe}, falls jedes $ a \in G $ ein inverses Element $ a^{-1} \in G $ besitzt.
\end{defn*}

\begin{exmp*}
	$ GL_n(\R) = \{A \in M_{n \times n} (\R) \mid \det(A) \neq 0, n \geq 0\} $ ist eine Gruppe unter Matrixmultiplikation. Für $ n \geq 2 $ gibt es Matrizen $ A,B \in GL_n(\R) $ mit $ AB \neq BA $.
\end{exmp*}

\begin{defn*}[abelsche Gruppe]\index{Gruppe!abelsche}
	Sei $ (G,\circ) $ eine Gruppe. $G$ heißt \emph{kommutativ} oder \emph{abelsch}, falls gilt $ a \circ b = b \circ a \ \foralll a,b \in G $
\end{defn*}